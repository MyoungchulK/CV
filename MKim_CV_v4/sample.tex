%%%%%%%%%%%%%%%%%
% This is an sample CV template created using altacv.cls
% (v1.7, 9 August 2023) written by LianTze Lim (liantze@gmail.com). Compiles with pdfLaTeX, XeLaTeX and LuaLaTeX.
%
%% It may be distributed and/or modified under the
%% conditions of the LaTeX Project Public License, either version 1.3
%% of this license or (at your option) any later version.
%% The latest version of this license is in
%%    http://www.latex-project.org/lppl.txt
%% and version 1.3 or later is part of all distributions of LaTeX
%% version 2003/12/01 or later.
%%%%%%%%%%%%%%%%

%% Use the "normalphoto" option if you want a normal photo instead of cropped to a circle
% \documentclass[10pt,a4paper,normalphoto]{altacv}

\documentclass[10pt,a4paper,ragged2e,withhyper]{altacv}
%% AltaCV uses the fontawesome5 and packages.
%% See http://texdoc.net/pkg/fontawesome5 for full list of symbols.

% Change the page layout if you need to
\geometry{left=1.25cm,right=1.25cm,top=1.5cm,bottom=1.5cm,columnsep=1.2cm}

% The paracol package lets you typeset columns of text in parallel
\usepackage{paracol}

% Change the font if you want to, depending on whether
% you're using pdflatex or xelatex/lualatex
% WHEN COMPILING WITH XELATEX PLEASE USE
% xelatex -shell-escape -output-driver="xdvipdfmx -z 0" sample.tex
\ifxetexorluatex
  % If using xelatex or lualatex:
  \setmainfont{Roboto Slab}
  \setsansfont{Lato}
  \renewcommand{\familydefault}{\sfdefault}
\else
  % If using pdflatex:
  \usepackage[rm]{roboto}
  \usepackage[defaultsans]{lato}
  % \usepackage{sourcesanspro}
  \renewcommand{\familydefault}{\sfdefault}
\fi

% Change the colours if you want to
\definecolor{SlateGrey}{HTML}{2E2E2E}
\definecolor{LightGrey}{HTML}{666666}
\definecolor{DarkPastelRed}{HTML}{450808}
\definecolor{PastelRed}{HTML}{8F0D0D}
\definecolor{GoldenEarth}{HTML}{E7D192}
\colorlet{name}{black}
\colorlet{tagline}{PastelRed}
\colorlet{heading}{DarkPastelRed}
\colorlet{headingrule}{GoldenEarth}
\colorlet{subheading}{PastelRed}
\colorlet{accent}{PastelRed}
\colorlet{emphasis}{SlateGrey}
\colorlet{body}{LightGrey}

% Change some fonts, if necessary
\renewcommand{\namefont}{\Huge\rmfamily\bfseries}
\renewcommand{\personalinfofont}{\footnotesize}
\renewcommand{\cvsectionfont}{\LARGE\rmfamily\bfseries}
\renewcommand{\cvsubsectionfont}{\large\bfseries}


% Change the bullets for itemize and rating marker
% for \cvskill if you want to
\renewcommand{\cvItemMarker}{{\small\textbullet}}
\renewcommand{\cvRatingMarker}{\faCircle}
% ...and the markers for the date/location for \cvevent
% \renewcommand{\cvDateMarker}{\faCalendar*[regular]}
% \renewcommand{\cvLocationMarker}{\faMapMarker*}


% If your CV/résumé is in a language other than English,
% then you probably want to change these so that when you
% copy-paste from the PDF or run pdftotext, the location
% and date marker icons for \cvevent will paste as correct
% translations. For example Spanish:
% \renewcommand{\locationname}{Ubicación}
% \renewcommand{\datename}{Fecha}


%% Use (and optionally edit if necessary) this .tex if you
%% want to use an author-year reference style like APA(6)
%% for your publication list
% % When using APA6 if you need more author names to be listed
% because you're e.g. the 12th author, add apamaxprtauth=12
\usepackage[backend=biber,style=apa6,sorting=ydnt]{biblatex}
\defbibheading{pubtype}{\cvsubsection{#1}}
\renewcommand{\bibsetup}{\vspace*{-\baselineskip}}
\AtEveryBibitem{%
  \makebox[\bibhang][l]{\itemmarker}%
  \iffieldundef{doi}{}{\clearfield{url}}%
}
\setlength{\bibitemsep}{0.25\baselineskip}
\setlength{\bibhang}{1.25em}


%% Use (and optionally edit if necessary) this .tex if you
%% want an originally numerical reference style like IEEE
%% for your publication list
\usepackage[backend=biber,style=ieee,sorting=ydnt,defernumbers=true]{biblatex}
%% For removing numbering entirely when using a numeric style
\setlength{\bibhang}{1.25em}
\DeclareFieldFormat{labelnumberwidth}{\makebox[\bibhang][l]{\itemmarker}}
\setlength{\biblabelsep}{0pt}
\defbibheading{pubtype}{\cvsubsection{#1}}
\renewcommand{\bibsetup}{\vspace*{-\baselineskip}}
\AtEveryBibitem{%
  \iffieldundef{doi}{}{\clearfield{url}}%
}


%% sample.bib contains your publications
\addbibresource{sample.bib}

\begin{document}
\name{Myoungchul Kim}
\tagline{Data Scientist}
%% You can add multiple photos on the left or right
%\photoR{2.8cm}{Globe_High}
% \photoL{2.5cm}{Yacht_High,Suitcase_High}

\personalinfo{%
  % Not all of these are required!
  \email{kmc8907@gmail.com}
  \phone{080-9801-7956}
  \location{Greater Tokyo, Japan}
  \github{MyoungchulK}
  \linkedin{myoungchulkim}
  %\homepage{Portfolio}
  %\orcid{0000-0002-8624-5564}
  %% You can add your own arbitrary detail with
  %% \printinfo{symbol}{detail}[optional hyperlink prefix]
  % \printinfo{\faPaw}{Hey ho!}[https://example.com/]

  %% Or you can declare your own field with
  %% \NewInfoFiled{fieldname}{symbol}[optional hyperlink prefix] and use it:
  % \NewInfoField{gitlab}{\faGitlab}[https://gitlab.com/]
  % \gitlab{your_id}
  %%
  %% For services and platforms like Mastodon where there isn't a
  %% straightforward relation between the user ID/nickname and the hyperlink,
  %% you can use \printinfo directly e.g.
  % \printinfo{\faMastodon}{@username@instace}[https://instance.url/@username]
  %% But if you absolutely want to create new dedicated info fields for
  %% such platforms, then use \NewInfoField* with a star:
  % \NewInfoField*{mastodon}{\faMastodon}
  %% then you can use \mastodon, with TWO arguments where the 2nd argument is
  %% the full hyperlink.
  % \mastodon{@username@instance}{https://instance.url/@username}
}

\makecvheader
%% Depending on your tastes, you may want to make fonts of itemize environments slightly smaller
% \AtBeginEnvironment{itemize}{\small}

%% Set the left/right column width ratio to 6:4.
\columnratio{0.6}

% Start a 2-column paracol. Both the left and right columns will automatically
% break across pages if things get too long.
\begin{paracol}{2}
\cvsection{Projects}

\cvevent{\href{https://user-web.icecube.wisc.edu/~mkim/ARA_MF/index.html}{Search for Ultra-high Energy Neutrinos from Askaryan Radio Array (ARA) by Template Method \faLink}}{International Center for Hadron Astrophysics (ICEHAP)}{March 2017 -- December 2023}{Chiba University}
\begin{itemize}
\item Classified astronomical signal by {\bf statistical-oriented Principal Component Analysis (PCA)}, after obtaining features from {\bf 2 billion amounts (${\sim}$200 TB)} of radio-frequency data that measured below the South Pole.
\item Implemented automation solutions for utilizing {\bf large CPU \& GPU clusters} by building {\bf Python \& C++ packages} to streamline data analysis workflows and enhance productivity.
\item Implemented physics techniques, such as the {\bf Fast Fourier Transform (FFT)}, {\bf Interferometry}, and the {\bf Matched Filter}, into the package for {\bf feature extraction}.
\item Optimized the PCA based on {\bf Frequentist Statistics} and {\bf Pseudo Experiment}.
\item Analyzed \& quantified result by calculating {\bf statistical significance}, including {\bf systematic uncertainty}, and {\bf Monte Carlo simulation}.

\end{itemize}

\divider

\cvevent{\href{https://github.com/MyoungchulK/Sound_To_Symphony_LeWgon}{Sound to Symphony (AI Music Generation) \faLink}}{Le Wagon Data Science \& AI Bootcamp}{January 2024 -- March 2024}{Le Wagon, Tokyo}
\begin{itemize}
\item Generates completely new music by {\bf Recurrent Neural Network (RNN)} that can be easily customizable by musical software
\item {\bf Architectures RNN model for learning musical patterns} from large classical music datasets that are expressed in numerical format. 
\item Deplyed the project into the {\bf Streamlit} by utilizing {\bf FastAPI}
\item Built the connection between generated music and musical software Abelton
\end{itemize}

\cvsection{Experience}

\cvevent{Graduate Research Assistant}{Chiba \& SungKyunKwan University}{2015 -- 2023}{Japan \& South Korea}
\begin{itemize}
\item Performed {\bf high-precision calibration for the radio-frequency antenna} for an advanced research instrument.
\item Established {\bf scientific Python \& C++ hybrid package}, inspired by C++-based code, that extracts physics results from raw data which has led to {\bf wildly use by international collaborators}.
\item Learned {\bf large database management}, including optimization of data sourcing and efficient connection to supercomputer by using solid analysis pipeline.
\item Practiced a thorough way to {\bf evaluate the project results by using statistical techniques} and the {\bf back of the envelope calculation}.
\end{itemize}
\divider

\cvevent{Teaching \& Operating Assistant}{Chiba \& SungKyunKwan University}{2015 -- 2018}{Japan \& South Korea}
\begin{itemize}
\item Guided Korean students are transitioning to undergraduate studies at Japanese universities by {\bf teaching freshman-level physics and mathematics courses in Japanese}.
\item Provided mentorship to students, aiding them in achieving academic success in challenging physics courses. Conducted hands-on physics experiments with students, enhancing their practical knowledge and skills. {\bf Managed scientific equipment} to ensure smooth and successful experiment execution.
\end{itemize}
\divider

\cvevent{Japanese to English \& Korean Translator}{Wovn.io}{2018}{Tokyo}
\begin{itemize}
\item Provided translation services and real-time deployment for the client company website written in Japanese, incorporating cultural nuances to enhance readability and accessibility.
\end{itemize}
\medskip

% use ONLY \newpage if you want to force a page break for
% ONLY the current column
%\newpage


\cvsection{Education}

\cvevent{Data Science \& AI Bootcamp}{Le Wagon}{January 2024 -- March 2024}{Japan}
\begin{itemize}
\item Thorough study in Python for Data Science, with expertise in data extraction, manipulation, and visualization, backed by a strong foundation in statistics and linear algebra.
\item Delving into Machine Learning and Deep Learning, with practical application in building comprehensive workflows utilizing Scikit-Learn and designing neural network architectures.
\item Proficiency in ML Engineering, involving the development of Python packages for large-scale data tasks in GCP, and a deep awareness of the ethical considerations surrounding AI deployment.
\end{itemize}
\divider

\cvevent{Completion of Ph.D. program, Physics}{Chiba University}{April 2017 -- December 2023}{Japan}
\begin{itemize}
\item Research topic: Search for Ultra-high Energy Neutrinos Using Eight Years of Data from Two ARA Stations by the Neutrino Template Method
\end{itemize}
\divider

\cvevent{Master of Science, Physics}{SungKyunKwan University}{March 2015 -- February 2017}{South Korea}
\begin{itemize}
\item Thesis title: Performance study of camera system for the IceCube-Gen2 detector
\end{itemize}
\divider

\cvevent{Bachelor in Science, Physics}{SungKyunKwan University}{March 2011 -- February 2015}{South Korea}

\switchcolumn

\cvsection{About Me}

\begin{quote}
I'm an Astrophysicist used to analyzing large datasets to find astronomical signals and recently graduated from Le Wagon Data Scientist \& AI boot camp. 

Seeking to utilize programming skills backed by my scientific background and data science knowledge.
\end{quote}

\cvsection{Technical Skills}

\cveventv{Coding Tools}
\cvtag{Python}
\cvtag{C++}
\cvtag{Vim}
\cvtag{Condor}\\
\cvtag{CVMFS}
\cvtag{G-Collab \& Jupyter}
\cvtag{Latex}

\dividerv\smallskip

\cveventv{Data Analytics}
\cvtag{NumPy}
\cvtag{SciPy}
\cvtag{Pandas}
\cvtag{SQL}\\
\cvtag{Matplotlib}
\cvtag{Seaborn}

\dividerv\smallskip

\cveventv{Modelling}
\cvtag{Ensemble Methods}
\cvtag{Langchain}\\
\cvtag{Statsmodels}

\dividerv\smallskip

\cveventv{Deployment}
\cvtag{GCP}
\cvtag{Docker}
\cvtag{FastAPI}
\cvtag{Streamlit}

\dividerv\smallskip

\cveventv{Hardware Experience}
\cvtag{Electronics}
\cvtag{Optics}

\cvsection{Languages}

\cvskillv{English}{Fluent}\\
\divider

\cvskillv{Japanese}{JLPT N1, Fluent}\\ %% Supports X.5 values.
\divider

\cvskillv{Korean}{Native}\\

%% Yeah I didn't spend too much time making all the
%% spacing consistent... sorry. Use \smallskip, \medskip,
%% \bigskip, \vspace etc to make adjustments.
\medskip

\cvsection{Interests}

\cvtag{Classical Music}\\
\cvtag{Orchestra}
\cvtag{Contrabass}\\
\cvtag{Universe}
\cvtag{Fourier Transform}

% Adapted from @Jake's answer from http://tex.stackexchange.com/a/82729/226
% \wheelchart{outer radius}{inner radius}{
% comma-separated list of value/text width/color/detail}
%\wheelchart{1.5cm}{0.5cm}{%
%  %8/8em/accent!60/Contrabass,
%  2/10em/accent/Orchestra,
%  5/6em/accent!20/Classical Music
%}

% \divider

\newpage

\cvsection{Awards}

\cvachievement{\faTrophy}{Japanese Government Monbukagakusho Scholarship (MEXT)}{Graduate Research Assistant in Ph.D., 2017 -- 2020}
\divider

\cvachievement{\faHeartbeat}{Teaching Assistant (T.A.) of Korea \& Japan Joint Government Scholarship}{Teaching Assistant for a freshman Korean students, 2017 -- 2018}
\divider

\cvachievement{\faTrophy}{BK21+ Research student scholarship}{Graduate Research Assistant in Msc., 2015 -- 2017}

\divider

\cvachievement{\faHeartbeat}{Operating Assistant scholarship}{Physics Experiment Assistant, 2016 -- 2017}

\divider

\cvachievement{\faTrophy}{CK Research student scholarship}{Research Assistant in Bsc., 2014}

\cvsection{Selected\\Publications}

%% Specify your last name(s) and first name(s) as given in the .bib to automatically bold your own name in the publications list.
%% One caveat: You need to write \bibnamedelima where there's a space in your name for this to work properly; or write \bibnamedelimi if you use initials in the .bib
%% You can specify multiple names, especially if you have changed your name or if you need to highlight multiple authors.
\mynames{Lim/Lian\bibnamedelima Tze,
  Wong/Lian\bibnamedelima Tze,
  Lim/Tracy,
  Lim/L.\bibnamedelimi T.}
%% MAKE SURE THERE IS NO SPACE AFTER THE FINAL NAME IN YOUR \mynames LIST

\nocite{*}

\printbibliography[heading=pubtype,title={\printinfo{\faFile*[regular]}{Journal Articles}},type=article]

\end{paracol}


\end{document}
